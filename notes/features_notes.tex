\documentclass[10pt,a4paper]{article}
\usepackage[utf8]{inputenc}
\usepackage{amsmath}
\usepackage{amsfonts}
\usepackage{amssymb}
\usepackage{makeidx}
\usepackage{graphicx}
\usepackage{lmodern}
\usepackage{kpfonts}
\usepackage[left=2cm,right=2cm,top=2cm,bottom=2cm]{geometry}
\usepackage{listings}

\title{Catalog of auxiliary visualization functions in rpart and randomForest}
\begin{document}
\maketitle
\pagenumbering{gobble}
\newpage
\tableofcontents
\newpage
\pagenumbering{arabic}
\section{rpart}
\subsection{Visualizing trees}
\bigskip
\paragraph*{plot.raprt}
\paragraph{plot.raprt}
\begin{lstlisting}[language=R]
plot.rpart <- function(x, uniform = FALSE, branch = 1, compress = FALSE,
                       nspace, margin = 0, minbranch = 0.3, ...)
{
    if (!inherits(x, "rpart")) stop("Not a legitimate \"rpart\" object")
    if (nrow(x$frame) <= 1L) stop("fit is not a tree, just a root")

    if (compress & missing(nspace)) nspace <- branch
    if (!compress) nspace <- -1L     # means no compression
    ## if (dev.cur() == 1L) dev.new() # not needed in R

    parms <- list(uniform = uniform, branch = branch, nspace = nspace,
                 minbranch = minbranch)

    ## define the plot region
    temp <- rpartco(x, parms)
    xx <- temp$x
    yy <- temp$y
    temp1 <- range(xx) + diff(range(xx)) * c(-margin, margin)
    temp2 <- range(yy) + diff(range(yy)) * c(-margin, margin)
    plot(temp1, temp2, type = "n", axes = FALSE, xlab = "", ylab = "", ...)
    ## Save information per device, once a new device is opened.
    assign(paste0("device", dev.cur()), parms, envir = rpart_env)

    # Draw a series of horseshoes or V's, left son, up, down to right son
    #   NA's in the vector cause lines() to "lift the pen"
    node <- as.numeric(row.names(x$frame))
    temp <- rpart.branch(xx, yy, node, branch)

    if (branch > 0) text(xx[1L], yy[1L], "|")
    lines(c(temp$x), c(temp$y))
    invisible(list(x = xx, y = yy))
}
\end{lstlisting}
\paragraph*{text.rpart}
\paragraph{text.rpart}
\begin{lstlisting}[language=R]
## This is a modification of text.tree.
## Fancy option has been added in (to mimic post.tree)
##

text.rpart <-
    function(x, splits = TRUE, label, FUN = text, all = FALSE,
             pretty = NULL, digits = getOption("digits") - 3L,
             use.n = FALSE, fancy = FALSE, fwidth = 0.8, fheight = 0.8,
             bg = par("bg"), minlength = 1L, ...)
{
    if (!inherits(x, "rpart")) stop("Not a legitimate \"rpart\" object")
    if (nrow(x$frame) <= 1L) stop("fit is not a tree, just a root")

    frame <- x$frame
    if (!missing(label)) warning("argument 'label' is no longer used")
    col <- names(frame)
    ylevels <- attr(x, "ylevels")
    if (!is.null(ylevels <- attr(x, "ylevels"))) col <- c(col, ylevels)
    cxy <- par("cxy")                   # character width and height
    if (!is.null(srt <- list(...)$srt) && srt == 90) cxy <- rev(cxy)
    xy <- rpartco(x)

    node <- as.numeric(row.names(frame))
    is.left <- (node %% 2L == 0L)            # left hand sons
    node.left <- node[is.left]
    parent <- match(node.left/2L, node)

    ## Put left splits at the parent node
    if (splits) {
        left.child <- match(2L * node, node)
        right.child <- match(node * 2L + 1L, node)
        rows <- if (!missing(pretty) && missing(minlength))
            labels(x, pretty = pretty) else labels(x, minlength = minlength)
        if (fancy) {
            ## put split labels on branches instead of nodes
            xytmp <- rpart.branch(x = xy$x, y = xy$y, node = node)
            leftptx <- (xytmp$x[2L, ] + xytmp$x[1L, ])/2
            leftpty <- (xytmp$y[2L, ] + xytmp$y[1L, ])/2
            rightptx <- (xytmp$x[3L, ] + xytmp$x[4L, ])/2
            rightpty <- (xytmp$y[3L, ] + xytmp$y[4L, ])/2

            FUN(leftptx, leftpty + 0.52 * cxy[2L],
                rows[left.child[!is.na(left.child)]], ...)
            FUN(rightptx, rightpty - 0.52 * cxy[2L],
                rows[right.child[!is.na(right.child)]], ...)
        } else
            FUN(xy$x, xy$y + 0.5 * cxy[2L], rows[left.child], ...)
    }

    leaves <- if (all) rep(TRUE, nrow(frame)) else frame$var == "<leaf>"

    stat <-
        x$functions$text(yval = if (is.null(frame$yval2)) frame$yval[leaves]
                                else frame$yval2[leaves, ],
                         dev = frame$dev[leaves], wt = frame$wt[leaves],
                         ylevel = ylevels, digits = digits,
                         n = frame$n[leaves], use.n = use.n)

    if (fancy) {
        if (col2rgb(bg, alpha = TRUE)[4L, 1L] < 255) bg <- "white"
        oval <- function(middlex, middley, a, b)
        {
            theta <- seq(0, 2 * pi, pi/30)
            newx <- middlex + a * cos(theta)
            newy <- middley + b * sin(theta)
            polygon(newx, newy, border = TRUE, col = bg)
        }

        ## FIXME: use rect()
        rectangle <- function(middlex, middley, a, b)
        {
            newx <- middlex + c(a, a, -a, -a)
            newy <- middley + c(b, -b, -b, b)
            polygon(newx, newy, border = TRUE, col = bg)
        }

        ## find maximum length of stat
        maxlen <- max(string.bounding.box(stat)$columns) + 1L
        maxht <- max(string.bounding.box(stat)$rows) + 1L

        a.length <- if (fwidth < 1)  fwidth * cxy[1L] * maxlen else fwidth * cxy[1L]

        b.length <- if (fheight < 1) fheight * cxy[2L] * maxht else fheight * cxy[2L]

        ## create ovals and rectangles here
        ## sqrt(2) creates the smallest oval that fits around the
        ## best fitting rectangle
        for (i in parent)
            oval(xy$x[i], xy$y[i], sqrt(2) * a.length/2, sqrt(2) * b.length/2)
        child <- match(node[frame$var == "<leaf>"], node)
        for (i in child)
            rectangle(xy$x[i], xy$y[i], a.length/2, b.length/2)
    }

    ##if FUN=text then adj=1 puts the split label to the left of the
    ##    split rather than centered
    ##Allow labels at all or just leaf nodes

    ## stick values on nodes
    if (fancy) FUN(xy$x[leaves], xy$y[leaves] + 0.5 * cxy[2L], stat, ...)
    else FUN(xy$x[leaves], xy$y[leaves] - 0.5 * cxy[2L], stat, adj = 0.5, ...)

    invisible()
}
\end{lstlisting}
\subsection{Variable importance}
\bigskip
\paragraph*{importance.rpart}
\paragraph{importance.rpart}
\begin{lstlisting}[language=R]
#
# Caclulate variable importance
# Each primary split is credited with the value of splits$improve
# Each surrogate split gets split$adj times the primary split's value
#
# Called only internally by rpart
#
importance <- function(fit)
{
    ff <- fit$frame
    fpri <- which(ff$var != "<leaf>")  # points to primary splits in ff
    spri <- 1 + cumsum(c(0, 1 + ff$ncompete[fpri] + ff$nsurrogate[fpri]))
    spri <- spri[seq_along(fpri)] # points to primaries in the splits matrix
    nsurr <- ff$nsurrogate[fpri]  # number of surrogates each has

    sname <- vector("list", length(fpri))
    sval <- sname

    ## The importance for primary splits needs to be scaled
    ## It was a printout choice for the anova method to list % improvement in
    ##  the sum of squares, an importance calculation needs the total SS.
    ## All the other methods report an unscaled change.
     scaled.imp <- if (fit$method == "anova")
        fit$splits[spri, "improve"] * ff$dev[fpri]
    else fit$splits[spri, "improve"]

    sdim <- rownames(fit$splits)
    for (i in seq_along(fpri)) {
        ## points to surrogates
        if (nsurr[i] > 0L) {
            indx <- spri[i] + ff$ncompete[fpri[i]] + seq_len(nsurr[i])
            sname[[i]] <- sdim[indx]
            sval[[i]] <- scaled.imp[i] * fit$splits[indx, "adj"]
        }
    }

    import <- tapply(c(scaled.imp, unlist(sval)),
                     c(as.character(ff$var[fpri]), unlist(sname)),
                     sum)
    sort(c(import), decreasing = TRUE) # a named vector
}
\end{lstlisting}         
\section{randomForest}
\subsection{Visualizing trees}
\bigskip
\paragraph*{plot.randomForest}
\paragraph{plot.randomForest}
\begin{lstlisting}[language=R]
plot.randomForest <- function(x, type="l", main=deparse(substitute(x)), ...) {
  if(x$type == "unsupervised")
    stop("No plot for unsupervised randomForest.")
  test <- !(is.null(x$test$mse) || is.null(x$test$err.rate))
  if(x$type == "regression") {
    err <- x$mse
    if(test) err <- cbind(err, x$test$mse)
  } else {
    err <- x$err.rate
    if(test) err <- cbind(err, x$test$err.rate)
  }
  if(test) {
    colnames(err) <- c("OOB", "Test")
    matplot(1:x$ntree, err, type = type, xlab="trees", ylab="Error",
            main=main, ...)
  } else {
    matplot(1:x$ntree, err, type = type, xlab="trees", ylab="Error",
            main=main, ...)
  }
  invisible(err)
}
\end{lstlisting}
\subsection{Variable importance}
\bigskip
\paragraph*{importance.randomForest}
\paragraph{importance.randomForest}
\begin{lstlisting}[language=R]
importance <- function(x, ...)  UseMethod("importance")

importance.default <- function(x, ...)
    stop("No method implemented for this class of object")

importance.randomForest <- function(x, type=NULL, class=NULL, scale=TRUE,
                                    ...) {
    if (!inherits(x, "randomForest"))
        stop("x is not of class randomForest")
    classRF <- x$type != "regression"
    hasImp <- !is.null(dim(x$importance)) || ncol(x$importance) == 1
    hasType <- !is.null(type)
    if (hasType && type == 1 && !hasImp)
        stop("That measure has not been computed")
    allImp <- is.null(type) && hasImp
    if (hasType) {
        if (!(type %in% 1:2)) stop("Wrong type specified")
        if (type == 2 && !is.null(class))
            stop("No class-specific measure for that type")
    }
    
    imp <- x$importance
    if (hasType && type == 2) {
        if (hasImp) imp <- imp[, ncol(imp), drop=FALSE]
    } else {
        if (scale) {
            SD <- x$importanceSD
            imp[, -ncol(imp)] <-
                imp[, -ncol(imp), drop=FALSE] /
                    ifelse(SD < .Machine$double.eps, 1, SD)
        }
        if (!allImp) {
            if (is.null(class)) {
                ## The average decrease in accuracy measure:
                imp <- imp[, ncol(imp) - 1, drop=FALSE]
            } else {
                whichCol <- if (classRF) match(class, colnames(imp)) else 1
                if (is.na(whichCol)) stop(paste("Class", class, "not found."))
                imp <- imp[, whichCol, drop=FALSE]
            }
        }
    }
    imp
}
\end{lstlisting}
\paragraph*{varImpPlot.randomForest}
\paragraph{varImpPlot.randomForest}
\begin{lstlisting}[language=R]
varImpPlot <- function(x, sort=TRUE,
                       n.var=min(30, nrow(x$importance)),
                       type=NULL, class=NULL, scale=TRUE, 
                       main=deparse(substitute(x)), ...) {
    if (!inherits(x, "randomForest"))
        stop("This function only works for objects of class `randomForest'")
    imp <- importance(x, class=class, scale=scale, type=type, ...)
    ## If there are more than two columns, just use the last two columns.
    if (ncol(imp) > 2) imp <- imp[, -(1:(ncol(imp) - 2))]
    nmeas <- ncol(imp)
    if (nmeas > 1) {
        op <- par(mfrow=c(1, 2), mar=c(4, 5, 4, 1), mgp=c(2, .8, 0),
                  oma=c(0, 0, 2, 0), no.readonly=TRUE)
        on.exit(par(op))
    }
    for (i in 1:nmeas) {
        ord <- if (sort) rev(order(imp[,i],
                                   decreasing=TRUE)[1:n.var]) else 1:n.var
        xmin <- if (colnames(imp)[i] %in%
                    c("IncNodePurity", "MeanDecreaseGini")) 0 else min(imp[ord, i])
        dotchart(imp[ord,i], xlab=colnames(imp)[i], ylab="",
                 main=if (nmeas == 1) main else NULL,
                 xlim=c(xmin, max(imp[,i])), ...)
    }
    if (nmeas > 1) mtext(outer=TRUE, side=3, text=main, cex=1.2)
    invisible(imp)
}
\end{lstlisting}
\end{document}